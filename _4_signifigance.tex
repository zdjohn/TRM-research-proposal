\section{Future Significance}

\subsection{Practical Significance}
Recommender system is becoming more and more prominent in people's daily time. Users are overwhelmed by information that is pushed by different entities. Identify users' interest to assists decision making is an important mission for recommender systems. Meanwhile, due to the nature difference among industry and businesses, implementing and productionize a recommender system often face data sparsity and cold start challenges.  

Knowledge graph based recommender system persists information objects within its semantic network structures. Such property allows higher-order connections between nodes can be learned and used into recommender systems' training, which effectively improves data density. Graph structure also enables new information to be added easier and organically comparing to structured data.  

Thus, knowledge graph based recommender system can provide a generic yet effective approach, that reduce “bring to live” cost for business implementing recommender systems. 

\subsection{Theoretical Significance}
Most of the recommender system research treats recommendation problem as a static snapshot. The consequences of such approach means the recommendation model's ability of adapting new data points are low. Data changes are rarely scoped as part of recommender system design, even though that is a key characterize of real-world datasets. 

For this study, we are focusing on recommender systems design by leverage knowledge graph structure. We explore the entity representation property inside heterogenous graph and its knowledge transferring ability under cross-domain settings to improve recommendation quality and model adaptiveness, especially, when initial datasets are sparse.


