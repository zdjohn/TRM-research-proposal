\section{Future Impact/Significance}

\subsection{Practical Significance Analysis}
Recommender system is becoming more and more prominent in people's daily time. Users are overwhelmed by information that is pushed by different entities. Identify users' interest to assists decision making is an important mission for recommender systems. Meanwhile, due to the nature difference among different industry and business, implementing and productionize a recommender system into real-world use, normally require big effort. An effective recommender system often heavily relied on domain knowledge and tailor-made solutions. What's more, most of the existing recommender system needs constant maintenance and adjustment.  

Collaborative Filtering based recommender system gains a lot of traction in the industry. To my view, that is not only because the model can perform under certain conditions, its popularity is also gained due to its engineering simplicity and being a generic to different kinds of industry and business background. Heterogeneous Information Network based recommender system persists information objects with semantic network structures. It allows new information to be adapted and reflected relatively easy by extending relationship between new information objects (nodes, edges).  

Thus, developing a generic HIN approach for recommender system could significantly simply engineering complexity and “bring to live” cost for business to be benefiting from having recommender system in place. 

\subsection{Theoretical Significance Analysis}
Research on Recommender System seldomly considers feature adaptability along with temporal information progression. Most of the recommender system research treats recommendation problem as a static snapshot of time. The consequences of such approach means the subtle change between time period is not being used for making recommendations. This also makes the assumptions that datasets would not change significantly over time, such as new important features would not merge in future period, even though constant change is a key characterize of real-world datasets. 

For this study, we are focusing adaptability for recommender systems by leverage HIN network structure.  In this way, this research will improve recommender systems ability in accommodating changes. 

The result will help making a long-lasting effective recommender system, while reduce the maintenance cost of recommendation models. 


