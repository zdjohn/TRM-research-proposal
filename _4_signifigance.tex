\section{Significance}

\subsection{Practical Significance}
This research develops an adaptive recommendation framework based on cross-domain knowledge graphs. Its result would significantly reduce the recommender system's "bring to market" cost and improve its business adoption.

In this research, we illustrated a systematic framework to overcome the recommendation cold start challenges under sparse data conditions. 
A holistic approach is introduced, that combines knowledge graph, representation learning, and knowledge transfer learning into an end-to-end deep learning system. 
The research result would greatly simplify engineering complexity, reduce cost, and improve performance for early-stage recommender systems in production. Consequently, Recommender systems would become more accessible to a broader business audience for real-world applications.


\subsection{Theoretical Significance}
This research develops an adaptive recommendation framework that could greatly improve the recommendation performance under sparse data and cold start conditions.

Inside the heterogeneous knowledge graph, user-item metadata, as well as user-item interactions, are natively preserved (connected) via nodes and edges within the graph structure.  We combine the knowledge graph based representation and cross-domain transfer learning technique into a unified framework to enrich and extract user-item information for improved data density and richer node representation. Further, its inductive representation ability means the training dataset would no longer limit the recommender system's prediction ability. 
As a result, the research outcome would make the recommendation model more adaptive to cold start problems and resistant to sparse dataset.
