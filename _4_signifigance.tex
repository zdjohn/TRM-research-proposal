\section{Future Significance}

\subsection{Practical Significance}
This research develops an adaptive recommendation framework based on cross-domain knowledge graphs. Its result would significantly reduce the recommender system's "bring to market" cost and improve its business adoption.

Data sparsity issues are big barriers in launching a real-world recommender system, which often leads to low prediction performance. 
The recommendation model is more prone to have cold start problems because of the limited training data. 
In this research, we illustrated a systematic framework to overcome the mentioned challenges. 
A holistic approach is introduced, that combines knowledge graph, representation learning, and knowledge transfer learning into an end-to-end schema. 
The research result would reduce engineering effort and cost for building generic recommender systems, consequently allowing more accessible business implementation.


\subsection{Theoretical Significance}
This research develops an adaptive recommendation framework that could greatly improves the recommendation performance under sparse data and cold start conditions.

Inside the heterogeneous knowledge graph, user-item metadata, as well as user-item interactions, are natively preserved (connected) via nodes and edges within the graph structure.  We combine the knowledge graph based representation and cross-domain transfer learning technique into a unified framework to enrich and extract user-item information for improved data density and richer node representation. Further, its inductive representation ability means the training dataset would no longer limit the recommender system's prediction ability. 
As a result, the research outcome would make the recommendation model more adaptive to cold start problems and resistant to sparse dataset.

% Research leverages graph based message propagation and aggregation ability

% Most of the recommender system research treats recommendation problem as a static snapshot. The consequences of such approach means the recommendation model's ability of adapting new data points are low. Data changes are rarely scoped as part of recommender system design, even though that is a key characterize of real-world datasets. 

% For this study, we are focusing on recommender systems design by leverage knowledge graph structure. We explore the entity representation property inside heterogeneous graph and its knowledge transferring ability under cross-domain settings to improve recommendation quality and model adaptiveness, especially, when initial datasets are sparse.


