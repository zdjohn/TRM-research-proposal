\section*{Summary}

\subsection*{Topic} Heterogeneous Information Network based Recommender System

\subsection*{Keywords} 

Recommender system, Heterogeneous Information Network, Knowledge Graph, Data Mining 

\subsection*{Thesis statement}
By learning semantic information of Heterogeneous Information Network can help in developing a generic approach for implementing recommender systems that is long lasting and easy on maintenance. 

\subsection*{Question 1}
How to develop a recommendation system that is adaptive to dynamically updating data streams?

\subsection*{Question 2}
How to use graph rich semantic information to overcome cold start problem in recommender systems?

\subsection*{Question 3}
How can we transfer transfer learnt knowledge across network domains for more effective recommendation results?

\subsection*{Objective 1}
To develop a formalized graph structure modelling methodology for rich feature data sets. 

\textbf{Success condition:} Load raw data set into a graph structure that can effective present the information required for recommendation problem accordingly 

\subsection*{Objective 2} 
To develop a comprehensive similarity method based on graph structure that can applied into recommender systems. 

\textbf{Success condition:} Evaluate current graph-based similarity measures to determine and develop a fitting similarity approach for recommender system, which is, capable of solving Top-K recommendation and effective in Cold Start problems. 

\subsection*{Objective 3}
To develop a semi-supervised feature mining method to handle complex large heterogeneous information network. 

\textbf{Success condition:} An embedding approach that is not only performant but also computational effective for mining large scale HIN. 

\subsection*{Objective 4} 
To develop a graph-based learning framework that adaptive to tempo information 

\textbf{Success condition:} A graph-based framework that is capable learning feature changes over time. 

\subsection*{Objective 5} 
To develop a recommendation system based on Heterogeneous Information Network 

\textbf{Success condition:} A working recommender framework prototype using generic approach that is capable to  adapting to feature change and computational effective.  

\newpage

\section{Aims}
\subsection{Aim 1}
To explore the semantic information relationship within Heterogeneous Information Network for Recommendation problem.

\subsection{Aim 2}
To understand the effectiveness and adaptability of Heterogeneous Information Network for constant data stream processing in recommendation problem domain. 

\subsection{Aim 3}
Try to develop a generic approach based Heterogeneous Information Network to make a long-lasting recommender system that is both adaptive to feature changes and being temporal aware. 

\section{Objectives}
This research aim to achieve following objectives: 

\subsection{To develop a formalized graph structure modelling methodology for rich feature data sets.  }

Develop a generic approach to project users, items, and their related feature information into a holistic heterogeneous Information Network. For Recommender System, HIN will not only need to handle handle complex structures. So that, both user-item interaction, commonly being used for collaborative filtering approach. User Profile, Item Attribute, which widely adopted in content based recommender system, can be fused in to single Multiple-Hub Network \citep{Shi2017} structure.

\subsection{To develop a comprehensive similarity method based on graph structure that can applied into recommender systems. } 

For the heterogenous information graph, evaluate the similarity of objects is the fundamental in data mining. In recommender system, it faces ever growing and evolving dataset. Working against massive datasets, and complex data transformation are common obstacles for producing a high quality recommendation solution. Developing an computational efficient similarity measure that fits for recommender system's use case, is a key milestone to ensure my research success.

\subsection{To develop a semi-supervised feature mining method to handle complex large heterogeneous information network.}

The rich information contained within the same HIN may only useful for certain recommendation problems. Nodes and path would have different weights when optimization objective or time is different. Developing a semi-supervised mining method would help recommender system to find different weights on based on the inter relationship between nodes and meta-paths. Reducing complex features into lower dimensions would also helping reduce the computation complex \citep{Cai2018} as mentioned in Object 2, Further simplify implementation difficulties in recommender systems. 

\subsection{To develop a graph-based learning framework that adaptive to tempo information}

Time factor is an important factor for making recommendations. Users past context could have direct impact of user current interest. \citet{Song2019} had shown that measurable improvement performance can be found in recommender system, when exploiting time information in the recommendation process. As a extension objective from Objective 2 and 3, including temporal dynamics and considering the transitive similarity between nodes and edges is another important factor in delivering quality results.

\subsection{To develop a recommendation framework based on Heterogeneous Information Network} 

By putting research results from Objective 1,2,3 into a holistic framework, would enable us to develop a generic approach and making recommendation model adapting changes could significantly reduce the recommendation systems maintenance cost and keeping system performance over time. 
Instead of focusing the recommendation research on static datasets. In this research, we are trying to solve the real-world recommendation problem in a dynamic angle.

\subsection{To develop a recommender system case study for validating the proposed approaches}

Following common datasets will be used for method verification: 

\begin{itemize}

\item MovieLens dataset (http://grouplens.org/datasets/movielens/) 

\item Netflix dataset (http://www.lifecrunch.biz/archives/207) 

\item DBLP Citation Networks (https://dblp.uni-trier.de)  

\end{itemize}

Case studies on real estate and tourism recommendation will be conducted to validate the user-interest drifts overtime, to show how graph-based approach adapts to the ever-evolving data changes. 
