%!TEX root = main.tex

\section{Research Question, Objectives, and Expected Outcome}

Recommender Systems (RS) predicts user-item interests by learning historical interaction records. Keeping a recommender system effective over time and adaptive to constant data updates is a challenging and costly task. For large systems, routinely retraining model is both economically expensive and time-consuming. As a result, finding a recommendation approach, that not only performs well but also easy to maintain, is becoming an essential aspect in RS real-world adoptions.
In this research, we aim to develop a recommender system by learning rich semantic information of knowledge graph. That is, adaptive, easy on maintenance, and capable of making recommendations in an environment that faces continuously changing data. 

\subsection{Questions}

\subsubsection*{Question 1}
How to present users and items information in a knowledge graph for recommendation?

\subsubsection*{Question 2}
How to effectively transfer knowledge to assist recommendation among multiple knowledge graphs?

\subsubsection*{Question 3}
How to develop a cross-domain recommender system that is adaptive to new data overtime?

\subsection{Objectives}
This research aim to achieve following objectives: 

\bigskip
\textbf{Objective 1:} To develop a graph based feature learning method for recommendation systems.(Aim to answer question 1)

Collaborative Filtering (CF) based recommendation systems commonly, only uses user-item interactions dataset in training. However, such approach is ignoring important user profile and item feature details. In hybrid approach, when combining CF with content based recommendation approach, feature engineering is a big challenge in terms of balancing importance across multiple user or item features. Such fusion approach also faces computation and memory management challenges. 

In this study, by leveraging the heterogeneous networks, user-item interactions, as well as users/items attributes, can be naturally projected into single Multiple-Hub Network \citep{Shi2017} structure. Those users/items and their relationship can be treated as nodes and edges in a generic knowledge graph. Nodes presentation then can be learned holistically by traversing the knowledge graph. As a result, similarities between nodes can be measure by both distance and neighborhood structure. Such approach would allow recommender systems to incorporate richer information for trainings and predictions.

\bigskip
\textbf{Objective 2:} To develop a recommendation approach that is capable of making recommendations on unseen datasets. (Aim to answer question 1 and 3)

Commonly, recommender systems use user/item latent feature for prediction, which were learned through a static datasets. For new item or user that appeared post training, their latent feature would not be known, until they were included in the retrain later.

In this study, we aim to develope a recommender system that is capable of making predictions on unseen data points, without require retaining. New entity presentation is calculated by incorporate the inductive property of knowledge graph.

\bigskip
\textbf{Objective 3:} To develop a framework that is capable of training with both interacted and non-interacted user/item jointly for recommendation (Aim to answer question 2 and 3)

The other challenge most recommender systems face is, its training dataset being limited by user-item interactions availability. Item or user do not have any interaction are excluded out of the training process. As a result, the recommendation model is unable to make predictions for those non-interacted users or items.
Knowledge graph provides multiple connection points to establish relationships between nodes beyond user-item interactions. Inside the graph, the node-node relationship then can be further propagated via different meta-path, which allows reducing bias from highly connected nodes. 

Here we aim to use the meta-path property to learn presentation on both interacted and non-interacted nodes through the same training process. User-item interaction knowledge is transferred via knowledge graph to those non-interacted node. So that the trained recommendation model can making prediction including non-interacted user/item.

\bigskip
\textbf{Objective 4:} To develop a learning method based on Graph Neural Networks that can transfer knowledge cross domains. (Aim to answer question 2)

Subsequent to object 2 and 3, some single domain graph suffers from data sparsity problems. In order to alleviate the issue,
our research try to use graph propagation to transfer knowledge from a more dense source to target domain for recommendation performance improvement.

% The source domain can either coming from a similar (but different) problem domain. 

% In the era of user generated content, new contents is added every second. Retraining recommendation systems with large volume datasets are not only costly but also impractical.


% \textbf{Success condition:} A graph-based framework that is capable learning feature changes over time. 
\bigskip
\textbf{Objective 5:} To develop a adaptive recommender system case study for validating the proposed approaches. (Aim to answer question 1,2,3)

By putting research results from Objective 1,2,3,4 into a holistic framework, would enable us to develop a generic approach and making recommendation model adapting to data changes. Such system could significantly reduce the recommendation systems maintenance cost and keeping system performance over time. 

Following common datasets will be used for method verification: 

\begin{itemize}

\item MovieLens dataset (http://grouplens.org/datasets/movielens/) 

\item Netflix dataset (http://www.lifecrunch.biz/archives/207) 

\item DBLP Citation Networks (https://dblp.uni-trier.de)  

\end{itemize}

Case studies will be conducted on real estate and tourism recommendations.
