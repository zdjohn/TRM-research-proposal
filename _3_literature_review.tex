\section{Literature Review}

Known for its Semitic properties, Heterogeneous Information Network captures information and relationship between different types of data points. Comparing with traditional column-based data structure, Its adaptability in every changing data context, is much more forgiving for dynamic data models. Nodes and Meta-Path information mining is playing an important role in the data mining research domain. In recommender systems, a lot of time, we are tasked to making recommendations based on similarity. Techniques such PathSim \citep{Sun2011PathSim} and HeteSim \citep{Shi2013HeteSim} provided a rich foundation for similarity measure in HIN. Its results can be naturally borrowed into recommender system in KNN settings and Top-K recommendations. 

In the feature learning space, node2vec (Grover and Leskovec 2016), an algorithmic framework for learning continuous feature representations for nodes in networks. In node2vec, nodes is mapped to low-dimensional space of features that maximizes the likelihood of preserving network neighborhoods of nodes. Using a biased random walk procedure to explores diverse neighborhoods, node2vec can learn task-independent representations in complex networks. GraphSAGE (Hamilton et al. 2018) furthered the graph feature learning to be inductive instead of requiring all nodes in the graph to be presented during training of the embeddings. This extend the graph generalization ability to unseen nodes. From recommender system perspective, allowed less dependency of background knowledge in the recommendation problem domain \citep{Hu2018}.  

Graph Classification using Structural Attention \citep{lee2018graph} is a good demonstration of attention-based learning techniques are applied in graph. Such feature learning ability makes a graph based data feature to be more versatile in facing data change and problem context switching. 
Techniques, such as, User-guided embedding, can be invaluable for catering to recommender system with ever changing data streams. Such approach can also effectively reduce the data noise problem by exploiting the signals residing in the data.  

Graph Neural Network (GNN) as an extension of deep learning approaches using graph data network structure have recently emerged. \citet{ying2018graph} shows promising signs of GNN being adopted in a large-scale deep recommendation engine. \citet{song2019session} propose a recommender system that model dynamic user behaviors and context-dependent social influence with a graph-attention neural network, which dynamically infers the influencers based on users’ current interests. Both of the research shown that GNN would be a promising approach for handling dynamic and temporal data in recommendation tasks.