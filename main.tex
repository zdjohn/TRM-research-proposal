\documentclass[12pt,a4 paper,title page]{article}
\usepackage[utf8]{inputenc}
\usepackage[british]{datetime2} % For urldate formatting
\usepackage{natbib}
\usepackage{graphicx}
\usepackage{float}
\usepackage[colorlinks = true,
            linkcolor = teal,
            urlcolor  = teal,
            citecolor = blue,
            anchorcolor = blue]{hyperref}
\usepackage{wrapfig}
\bibliographystyle{uts} % Import Harvard UTS style
\setcitestyle{aysep={}} % Remove comma separation between author and year for in-text citations
\title{\textbf{Research Proposal: Heterogeneous Information Network based Recommender System}}
\author{\large\textbf{Di Zhang}, \\
\textbf{supervisor: Professor Guanhua Zhang}, \\
\textbf{co-supervisor: Professor Jie Lu}, \\
\textbf{Student number: 31292711}}

\date{\Large{\textbf{October 2019}}}

\begin{document}

\maketitle
\hfill
\hfill

\section*{Summary}

\subsection{Topic} Heterogeneous Information Network based Recommender System

\subsection{Keywords}
Recommender system, Heterogeneous Information Network, Knowledge Graph, Data Mining 

\subsection{Thesis statement 1}
The Embedding based information mining from Heterogeneous Information Graph would both reduce recommendation system productization complexity and enables model to be self-evolving over time

\subsection{Thesis statement 2}
By learning semantic information of meta-path from Heterogeneous Network can help in developing a generic and self-evolving approach for implementing real-world recommender systems  

\subsection{Aim 1}
To explore the semantic information relationship within Heterogeneous Information Network for Recommendation problem.

\subsection{Aim 2}
To understand the effectiveness and adaptability of Heterogeneous Information Network for changing data streams over time. 

\subsection{Aim 3}
Try to develop a generic approach based Heterogeneous Information Network to make a long-lasting recommender system that is adaptive to feature changes and being temporal aware. 

\subsection{Objective 1}
To develop a formalised graph structure modelling methodology for rich feature data sets. 

\textbf{Success condition:} Load raw data set into a graph structure that can effective present the information required for recommendation problem accordingly 

\subsection{Objective 2} 
To develop a comprehensive similarity method based on graph structure that can applied into recommender systems. 

\textbf{Success condition:} Evaluate current state of art graph-based similarity measures to determine and develop a fitting similarity approach for Similarity Comparison between nodes and meta-path. 

\subsection{Objective 3}
To develop a semi-supervised feature mining method to handle complex large heterogeneous information network. 

\textbf{Success condition:} An embedding approach that is not only effective but also computational effective for mining large scale HIN. 

\subsection{Objective 4} 
To develop a graph-based learning framework that adaptive to tempo information 

\textbf{Success condition:} A graph-based framework that is capable learning feature changes over time. 

\subsection{Objective 5} 
To develop a recommendation system based on Heterogeneous Information Network 

\textbf{Success condition:} A working recommender framework prototype using generic approach that is capable to  adapting to feature change while easy to productionize.  


\section{Introduction and background}
Recommender System (RS) is an indispensable technology in this big data era. It evolves around our everyday life on multiple fronts. From daily curated news feed, to online shopping portal, to music playlist that we listen, and movies we watch. RS help us to find personalised interests from the ever-increasing information overloads. It is also acting as a decision-making assisting agent, that makes peoples life more efficient and focused. 

2006, 100 million datasets were released by Netflix (Bennett and Lanning 2007) This action expedited the research on Collaborative Filtering. Over decades, Collaborative Filtering is still at the forefront of Recommender systems Field. Many new approaches and extensions had been developed collaboration with Collaborative Filtering to tackle its cold start and data sparsity problems. Barkan and Koenigstein (2016) introduced item2vec for Collaborative Filtering also showed how different forms of recommendations can be used within Collaborative Filtering domain. 

On the deep-learning camp, recommender system is also striving by utilising the learning ability of Neural Network. Cheng et al. (2016) proposed `Wide and deep learning for recommendation system` to achieve both memorisation and generalisation in recommendation for Google Play. Covington (2016) demonstrated how deep neural network is implemented in YouTube, which helped billions of users discover their personalised content. Karatzoglou et al. (2013) proposed use Reinforcement Learning techniques personalised ranking based on users’ interests.  

Cross Domain recommender system is another promising approach which make recommendation across different item domains. It is especially useful when initial user interaction data is scarce. Elkahky et al. (2015) experimented this approach on multiple Microsoft products, and concluded multi domain recommendation system significantly outperforms single domain recommendation systems. 

However, a good working recommendation system is not only about recommendation algorithm, it is a systematic approach.  Collaborative Filtering being overwhelmingly popular. Its success is partly due to its effectiveness, but its simplicity and low data engineering requirement and domain knowledge dependency, made it one of the most accessible recommendation approaches for a lot of recommendation problem. Though Its struggle with data sparsity and cold start is a known issue to Collaborative Filtering 

Based on my research the challenges of setting up a successful real-world recommender system mainly comes from 3 parts:  
\begin{itemize}
\item Algorithm. A fitting algorithm to given recommendation problem. Setting up an effective recommendation system is different for every business and every problem. 

\item Data Engineering. Researcher and developers normally facing enormous yet complex datasets, in which, only small percentage can be used for effective recommendation algorithm. In many instances, data enrichment not only introduce valuable features, but inevitably introducing noises alone the way.  

\item Data Adaptation. In the real-world scenario, information inflow as streams. Features changes as time go by along with its importance to our recommendation objective. While modern data warehouse solutions, enabling us hoarding data much easier. Mining relevant information effectively and efficiently remain challenging. 
\end{itemize}
We live in a world of information, the information objects or data points around us are mostly interconnected with each other as a complicated network. Heterogenous Network, in a lot of time can serve a reflection of real-world information without loss of generality. World wide web, biology networks, as well as traffic system, etc. can all form into an information network. Based on existing research, Heterogeneous Information Graph/Network had shown a lot of promises in the data mine field. Several of promising research progress has had been achieved in the Heterogeneous Network space. 

Heterogeneous Information Network is known for its Semitic properties that captures information and relationship between different types of data points. It is also known for its adaptability in every changing data context. Comparing with traditional column-based data structure it is much more forgiving when data context or perceived data model changing its model structure. Vertices and Meta-Path information mining is playing an important role in the data mining research domain. In recommender systems, a lot of time, we are tasked to making recommendations based on similarity. Techniques such PathSim(2011) and HeteSim(2013) for heterogeneous networks relevance search problem can be naturally borrowed into recommender system based on relatedness of heterogeneous objects. 

In the feature learning space, node2vec (Grover and Leskovec 2016), an algorithmic framework for learning continuous feature representations for nodes in networks. In node2vec, nodes is mapped to low-dimensional space of features that maximizes the likelihood of preserving network neighbourhoods of nodes. Using a biased random walk procedure to explores diverse neighbourhoods, node2vec can learn task-independent representations in complex networks. GraphSAGE (Hamilton et al. 2018) furthered the graph feature learning to be inductive instead of requiring all nodes in the graph to be presented during training of the embeddings. This extend the graph generalisation ability to unseen nodes. From recommender system perspective, allowed less dependency of background knowledge in the recommendation problem domain.  

Graph Classification using Structural Attention (Lee et al.) is a good demonstration of attention-based learning techniques are applied in graph. Such feature learning ability makes a graph based general purpose data lake more versatile in serving more than one problem context. With User-guided embedding techniques can be invaluable for catering to a broad spectrum of user guidance evidenced by different expected clustering results. Such practice can also effectively reduce the data noise problem by exploiting the signals residing in the data.  

Graph Neural Network GNNs as an extension of deep learning approaches for graph data have recently emerged (TBC)

\section{Aims}
\subsection{Aim 1}
To explore the semantic information relationship within Heterogeneous Information Network for Recommendation problem.

\subsection{Aim 2}
To understand the effectiveness and adaptability of Heterogeneous Information Network for changing data streams over time. 

\subsection{Aim 3}
Try to develop a generic approach based Heterogeneous Information Network to make a long-lasting recommender system that is adaptive to feature changes and being temporal aware. 

\section{Objectives}
This research aim to achieve following objectives: 

\subsection{Objective 1: To develop a formalised graph structure modelling methodology for rich feature data sets.  }

Develop a generic approach to project users, items, and their related feature information into a holistic heterogeneous Information graph. Via the graph structure data, the semantic relation between data points (i.e. user, item, user profile, item feature, etc.) will be specified.

\subsection{Objective 2: To develop a comprehensive similarity method based on graph structure that can applied into recommender systems. } 

A computational efficient similarity method that be used within graph data structure, which can learn nodes similarity based on network features. For the heterogenous information graph, similarity can be learned based on the network structure itself, such as, node-node distance, meta-path propagation, as well as common paths (or node) shared between two vertices. Developing an efficient similarity measure, that is capable to run on massive datasets while maintain accuracy is important for recommender system. 

\subsection{Objective 3: To develop a semi-supervised feature mining method to handle complex large heterogeneous information network.}

The rich information contained within the same HIN may only useful for certain recommendation problems. Nodes and path would have different importance when optimisation objective is different. Developing a semi-supervised mining method would help recommender system to find different weights on based on the inter relationship between nodes and meta-paths. 

\subsection{Objective 4: To develop a graph-based learning framework that adaptive to tempo information } 
Time factor is an important factor for making recommendations. Users past context could have direct impact of user current interest. Research have shown that time is an important context information in recommendation. when exploiting time information in the recommendation process, we find a measurable improvement of recommender systems’ performance. 

\subsection{Objective 5: To develop a recommendation system based on Heterogeneous Information Network} 

Majority of the recommendation research are focused on static dataset. However, in real-world setting, new information is constantly feeding into the data warehouse. Developing a generic approach and making recommendation model adapting changes could significantly reduce the recommendation systems maintenance cost and keeping system performance over time. 

\subsection{Objective 6: To develop a recommender system case study for validating the proposed approaches}

Following common datasets will be used for method verification: 

such as MovieLens dataset (http://grouplens.org/datasets/movielens/) 

Netflix dataset (http://www.lifecrunch.biz/archives/207) 

DBLP Citation Networks (https://dblp.uni-trier.de)  

And case studies on real estate recommendation or tourism recommendation will be conducted to validate the user-interest drifts overtime, and how graph-based approach adapts to the ever-evolving changes. 

\section{Methodology}

Task 1: identify research issues 

Task 2: propose a framework to distil user/item features from heterogeneous information Network 

Task 3: Identify computation complexity in graph mining 

Task 4: develop a holistic HIN for user-item representation for recommender system 

Task 5: extend HIN user-item representation to accommodate temporary information through time 

Task 6: develop recommendation framework based HIN that is adaptable to continuous data stream overtime. 

Task 7:  A case study for validating the proposed recommendation approaches 



\section{Future Impact/Significance}

\subsection{Practical Significance Analysis}
Recommender system is becoming more and more prominent in people's daily time. Users are overwhelmed by information that is pushed by different entities. Identify users' interest to assists decision making is an important mission for recommender systems. Meanwhile, due to the nature difference among different industry and business, implementing and productionize a recommender system into real-world use, normally require big effort. An effective recommender system often heavily relied on domain knowledge and tailor-made solutions. What's more most of the existing recommender system needs constant maintenance and adjustment.  

Collaborative Filtering based recommender system gains a lot of traction in the industry. To my view, that is not only because the model can perform under certain conditions, its popularity is also gained due to its engineering simplicity and being a generic to different kinds of industry and business background. Heterogeneous Information Network based recommender system persists information objects with semantic network structures. It allows new information to be adapted and reflected relatively easy by extending relationship between new information objects (nodes, edges).  

Thus, developing a generic approach for recommender system could significantly simply engineering complexity and “bring to live” cost for business which can benefits from having recommender system in place. 

\subsection{Theoretical Significance Analysis}
Research on Recommender System seldomly considers feature adaptability a long with temporal information progression. Most of the recommender system research treats recommendation problem as a static snapshot of a time period. The consequences of such approach mean the subtle change between time period is not being used for making recommendations. This also makes the assumptions that datasets would not change significantly over time, such as new important features would not merge in future period, even though constant change is a key characterise of real-world datasets. For this study, we are focusing adaptability for recommender systems by leverage HIN network structure.  In this way, this research will improve recommender systems ability in accommodating change. The result will help making a long-lasting effective recommender system, while reduce the maintenance cost recommendation models. 

\clearpage

\bibliography{bibliography}

\end{document}
