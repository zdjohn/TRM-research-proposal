\section*{Abstract}
Recommender systems are in high demand across various industries due to its effectiveness and profits it brought to e-commerce, entertainment, advertising, and many other more businesses. However, the most challenging issue that impairs recommendation quality is data sparsity and cold start problems, which is caused by the large number of items compared with the limited number of consumption a user can make. Although some researches were devoted to developing algorithms in tackling these issues, we have not yet seen a framework that solves the data and model capacity challenges in a holistic approach. In this research, we aim to take combined advantage of graph-based user/item representation learning and cross-domain transfer learning techniques on top of heterogeneous knowledge graph to form a unified recommendation framework. Such a framework both enhances data density and model prediction capability within a consolidated end-to-end learning system. The research aims significantly improve the recommendation model performance in sparse or foreign data conditions to ease the barrier of launching recommender systems, hence promote its business adoption to a broader audience. In this research: 1) A representation learning method will be developed to enhance users and items embeddings in a unified heterogeneous knowledge graph for improving data density; 2) An adaptive strategy for recommendation method will be developed to improve recommender systems’ adaptiveness on new and unseen data with knowledge graph embedding. 3) A cross-domain recommendation method will be developed to transfer knowledge from source domains with relatively dense data to  enhanced recommendation in a sparse target domain. 4)Finally, a recommender system prototype will be implemented and a case study will be conducted to validate proposed methods. 

\subsection*{Keywords} 
Recommender System, Heterogeneous Information Network, Knowledge Graph, Transfer Learning.