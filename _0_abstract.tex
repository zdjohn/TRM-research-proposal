\section*{Abstract}
Recommender systems are in high demand across various industries. Its effectiveness has proven in different domains, such as e-commerce, entertainment, and advertising, etc. On the other hand, recommender systems are, still, a mostly exclusive privilege to large corporations. Data availability and model adaptiveness are the main challenges of bringing a recommender system to live. Although lots of research effort has been devoted to developing algorithms in tackling data sparsity and cold start problems. We have not yet seen a systematic approach of bootstrapping recommender systems for real-world use, especially under the condition, when data availability is scarce. In this research, we propose to use transfer learning to enhance data quality from external domains. Then, in combination, we leverage the rich entity representation of the knowledge graph for recommendation tasks. This study aims to ease the barrier of launching recommender systems, hence promote its adoption to a broader audience. In this research: 1)Propose a knowledge graph based framework to combine both user/item interactions and side information. 2)After that, a representation method in heterogeneous knowledge graph will be built for enhanced recommendation model training. 3)An extended recommendation method will be developed that allow the recommender system to be adaptive to unseen data points 4)A knowledge graph based transfer learning method will be introduced to further improve target domain data by sourcing external domain's richer datasets. 5) Finally, a recommender system case study to validate and test the above methods will be conducted.

\subsection*{Keywords} 
Recommender System, Heterogeneous Information Network, Knowledge Graph, Transfer Learning. 