%!TEX root = main.tex

\section{Introduction and background}
Recommender system (RS) is an indispensable technology, it evolves around our everyday life on multiple fronts. Recommender system help us to find personalized interests from information overloads \citep{Lu2015}. It can act as a decision assisting agents to simplify or accelerate decision making process. Further more, as we experiencing the exponential growth of data content, RS keeps us updated on new emerging information.
Over the years, we see RS being adopted across different industries. At the same time, we noticed that, bringing a RS into real-world application often faces many different challenges. 

Recommender systems exists long before computer was invented. Catalogs can be considered a one of the most primitive content-based recommender systems. In 2006, 100 million datasets were released by Netflix \citep{Bennett2007} The action expedited the research on collaborative filtering (CF). Over a decade, matrix factorization based approach is still at the forefront of recommender systems field. Being overwhelmingly popular, its success is not only due to its effectiveness, but also thanks to its simplicity \citep{Amatriain2016}, despite of having known issue with data sparsity and cold start problems. Many new approaches and extensions had been evolved based on collaborative filtering to tackle its computation challenges and data sparsity problems. 

Transfer Learning \citep{Pan2010} in cross-domain recommender system is another promising area which allowing recommendation possible in when initial user interaction data is scarce in target domains. \citet{Elkahky2015} experimented this approach on multiple Microsoft products, and concluded multi domain recommender system significantly outperforms single domain recommender systems. 

% Based on research observation, bringing a new recommender system to market need to overcome a number challenges. Such as, How to maximize information utilization from available dataset? How to make recommendation model prediction ability less depended on training data? How to enrich exiting data from a available external domain source. Besides algorithm and statistical analytics, a systematic approach is required for handling the limited data resources in the early stages of recommender system setting up . 

Based on research observation, when bringing a new recommender systems into production, the project is commonly challenged by many aspects. One common early stage challenge is data sparsity. Which, consequently, makes the recommender systems more prone to cold start problems, resulting sub-optimal prediction, and frequent model retrain for adjusting to un-trained data input. In order to ensure a successful project launch, there are 2 areas requires research attention. First, feature enrichment approaches, especially under the condition when data resources is limited. Second, designing a adaptive recommender systems, that is more accommodating to new information, and resilient to cold start scenario. 

In this research, we consider knowledge graph as a heterogeneous information network. Living in a world of information, the data entities around us are mostly interconnected with each other and be formed into an information network. Recent development from \citet{qin2020survey}, \citet{wang2018ripplenet}, and \citet{xi2020graph} have shown a lot of potential from the knowledge graph, which can help in developing generic recommendation framework for solving challenges stated above. Besides algorithm and statistical analytics, our aim is to propose a unified single end-to-end framework to ease the barrier of recommender system launch.


This study will focusing on leverage knowledge graph to improve recommender systems prediction capability and results. The rest of this report is organized as follows: 
First we go through research questions, objectives and expected outcomes in Section 2. Next, in section 3, we present an extensive literature review of the related work of this study. It will cover recent recommender systems, knowledge graph representation and transfer learning research results. Section 4 describes the significance and innovation of this research. Section 5 presents the methodology to conduct this research. Section 6 is about the ethics and risk consideration. Finally, Section 7 outlines the entire timeline of this research and reports on the research progress up-to-date.


