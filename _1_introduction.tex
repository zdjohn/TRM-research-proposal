%!TEX root = main.tex

\section*{Abstracts}
T.B.D.

\subsection*{Keywords} 
Recommender system, Heterogeneous Information Network, Knowledge Graph, Transfer Learning, Feature Representation Learning

\section{Introduction and background}
Recommender System (RS) is an indispensable technology in this big data era. It evolves around our everyday life on multiple fronts. 
From daily curated news feed, to online shopping portals, to music and movies we play. RS help us to find personalized interests from the ever-increasing information overloads \citep{Lu2015}. 
For complex problems, RS is capable of taking account of multiple factors to simplify decision making efforts, acting as a decision-making assisting agent, such as b2b recommendation \citep{shambour2012trust} and medical decision assisting \citep{zhang2017idoctor}.

% \subsection{Research landscape}
recommendation Systems exists long before computer was invented. Catalogs can be considered a one of the most primitive Content Based recommendation systems. In 2006, 100 million datasets were released by Netflix \citep{Bennett2007} The action expedited the research on Collaborative Filtering. Over decades, Matrix Factorization based approach is still at the forefront of Recommender systems Field. Many new approaches and extensions had been evolved based on Collaborative Filtering to tackle its computation challenges and data sparsity problems. 

On the deep-learning camp, recommender system is also striving by utilizing the learning ability of Neural Network. \citet{Cheng2016} proposed `Wide and deep learning for recommendation system` to achieve both memorization and generalization in recommendation for Google Play. \citet{Covington2016} demonstrated how deep neural network is implemented in YouTube, which helped billions of users discover their personalized content. \citet{Karatzoglou2013} proposed use Reinforcement Learning techniques personalized ranking based on users’ interests.  

Transfer Learning \citep{Pan2010} in Cross Domain recommender system is another promising area which allowing recommendation possible in when initial user interaction data is scarce in target domains. \citet{Elkahky2015} experimented this approach on multiple Microsoft products, and concluded multi domain recommendation system significantly outperforms single domain recommendation systems. 

% \subsection{Real world Challenge}

Building recommendation system requires systematic approach. Being overwhelmingly popular, Collaborative Filtering success is not only due to its effectiveness, but also thanks to its simplicity and low barrier of entry \citep{Amatriain2016}, despite of having known issue with data sparsity and cold start problems.

Based on research observation, the challenges of setting up a successful real-world recommender system mainly comes down to 3 parts:  

\begin{itemize}
\item Algorithm. Setting up an effective recommendation system is different for every business and every problem. A algorithm in recommender system is very hard to kept generic. Thus, transferability and reusability is low.

\item Data Adaptation. In the real-world scenario, information in flow as data streams. Features changes as time go by, along with recommendation objective. We often see diminishing performance from a developed recommender system over time. The ability to adapt data dynamics and making use of Temporal Information remain challenging in recommendation field.

\item Feature Engineering. Researcher and developers normally facing enormous yet complex datasets, in which, only small percentage can be effectively utilized for recommendation learning. In many instances, big data brings not only valuable features information for learning tasks, but also, inevitably introducing noises alone the way. A generic yet effective Feature Engineering methodology is a key building block for recommender system.
\end{itemize}

Living in a world of information, the information objects or data points around us are mostly interconnected with each other as a Heterogenous Network. World wide web, biology networks, as well as traffic system, etc. can all be formed into an information network. Heterogeneous Information Network had shown a lot of promising potential in developing generic recommendation framework for solving challenges above.

% \subsection{Paper structure}


