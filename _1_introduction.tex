%!TEX root = main.tex

\section*{Abstracts}
Recommender systems are in high demand across various industries. Its effectiveness has proven in different domains, such as e-commerce, entertainment, and advertising, etc. On the other hand, recommender systems are, still, a mostly exclusive privilege to large corporations. Data availability and model adaptiveness are the main challenges of bringing a recommender system to live. Although lots of research effort has been devoted to developing algorithms in tackling data sparsity and cold start problems. We have not yet seen a systematic approach of bootstrapping recommender systems for real-world use, especially under the condition, when data availability is scarce. In this research, we propose to use transfer learning to enhance data quality from external domains. Then, in combination, we leverage the rich entity representation of the knowledge graph for recommendation tasks. This study aims to ease the barrier of launching recommender systems, hence promote its adoption to a broader audience. In this research, 1) propose a knowledge graph based approach to combine both user/item interactions and side information. 2) After that, a representation method in heterogeneous knowledge graph will be built for enhanced recommendation model training. 3) an extended recommendation method will be developed that allow the recommender system to be adaptive to unseen data points 4)A knowledge graph based transfer learning method will be introduced to further improve target domain data by sourcing external domain's richer datasets. 5) Finally, a recommender system case study to validate and test the above methods will be conducted.

\subsection*{Keywords} 
Recommender system, Heterogeneous Information Network, Knowledge Graph, Transfer Learning, Feature Representation Learning

\section{Introduction and background}
Recommender System (RS) is an indispensable technology, it evolves around our everyday life on multiple fronts. RS help us to find personalized interests from information overloads \citep{Lu2015}. It can act as a decision assisting agents to simplify or accelerate decision making process. Further more, as we experiencing the exponential growth of data content, RS keeps us updated on new emerging information.
Over the years, we see RS being adopted across different industries. At the same time, we noticed that, bringing a recommender systems into real-world application often faces many different sorts of challenges. 

Recommender systems exists long before computer was invented. Catalogs can be considered a one of the most primitive content-based recommender systems. In 2006, 100 million datasets were released by Netflix \citep{Bennett2007} The action expedited the research on Collaborative Filtering (CF). Over decades, Matrix Factorization based approach is still at the forefront of Recommender systems Field. Many new approaches and extensions had been evolved based on Collaborative Filtering to tackle its computation challenges and data sparsity problems. 

Transfer Learning \citep{Pan2010} in Cross Domain recommender system is another promising area which allowing recommendation possible in when initial user interaction data is scarce in target domains. \citet{Elkahky2015} experimented this approach on multiple Microsoft products, and concluded multi domain recommender system significantly outperforms single domain recommender systems. 

Meanwhile, building recommender system requires systematic approach, beyond algorithm and statistical analytics. Being overwhelmingly popular, CF's success is not only due to its effectiveness, but also thanks to its simplicity and low barrier of entry \citep{Amatriain2016}, despite of having known issue with data sparsity and cold start problems. Based on research observation, the challenges of setting up a successful real-world recommender system mainly comes down to 3 parts:  

\begin{itemize}
\item Algorithm. Setting up an effective recommender system is different for every business and every problem. A algorithm in recommender system is very hard to kept generic. Thus, transferability and reusability is low.

\item Data Adaptation. In the real-world scenario, information in flow as data streams. Features changes as time go by, along with recommendation objective. We often see diminishing performance from a developed recommender system over time. The ability to adapt data and making use of new emerging information remain challenging in recommendation field.

\item Feature Representation. Researcher and developers normally facing enormous yet complex datasets, in which, only small percentage can be effectively utilized. A generic yet effective learning methodology is a key building block to overcome data sparsity and cold start problems for recommender systems.
\end{itemize}

Living in a world of information, the information objects or data points around us are mostly interconnected with each other as a Heterogenous Network. World wide web, biology networks, as well as traffic system, etc. can all be formed into an information network. In our research, we consider knowledge graph as heterogeneous information network having a lot of promising potential in developing generic recommendation framework for solving challenges above.

This study will focusing on leverage knowledge graph to improve early stages recommender systems prediction capability and accuracy. The rest of this report is organized as follows: 
First we go through research questions, objectives and expected outcomes in Section 2. Next, in section 3, we present an extensive literature review of the related work of this study. It will cover recent recommender systems, knowledge graph representation and transfer learning research results. Section 4 describes the significance and innovation of this research. Section 5 presents the methodology to conduct this research. Section 6 is about the ethics and risk consideration. Finally, Section 7 outlines the entire timeline of this research and reports on the research progress up-to-date.


