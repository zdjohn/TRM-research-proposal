%!TEX root = main.tex

\section{Introduction and background}
The recommender system is an indispensable technology. It frees us from information overloads (Lu et al. 2015), keeping us focused on information that matters to us the most. Recommender system can also act as decision assisting agents to simplify or accelerate the decision-making process. Over the years, we see the recommender system adoption across different industries. In particular, the collaborative filtering based approach recommendation algorithm has been at the forefront of recommender systems field. The success of collaborative filtering is not only due to its effectiveness. Its simplicity and generic methodology are much appreciated by the industries, despite having known limitation with computation and high requirements on data quality \citep{Amatriain2016}. Data sparsity and constant cold start are common challenges that business face when bringing the recommender system into real-world applications. 

The knowledge graph is one type of heterogeneous information network. Its ability to retrain rich semantic information has been long studied in the data mining researches \citep{Song2019} extensively. By utilizing such characteristics, allows the recommender system to consolidate both items/users side information and user-item interaction into a single heterogeneous knowledge graph. The network structure and its meta path can effectively improve data density when user-item interactions are sparse. Studies from \citet{qin2020survey}, \citet{wang2018ripplenet}, and \citet{xi2020graph} are good examples for showcasing knowledge graph's ability in developing generic frameworks for recommender systems.

The dynamic nature is another distinct character of the knowledge graph, as it can grow and evolve while accepting new pieces of information.  On the other hand, many knowledge graph based recommender systems is still using static graphs for training and inference. Such static recommendation approach is prone to cold start problems and struggles when working with changing data. Representation method, such as GraphSAGE \citep{hamilton2017inductive} and Graph Attention Network \citep{velivckovic2017graph}, had shown good progress on generating node embeddings for unseen data. However, its application in recommender systems is yet to be explored further.

The other approach of improving recommender systems performance is, enriching information to available dataset. Transfer learning \citep{Pan2010} techniques can enhance data quality by sourcing external-domains. \citet{Elkahky2015} experimented transfer learning approach on multiple Microsoft products, and concluded the multi-domains recommender system significantly outperforms single domain recommender systems. Knowledge graph has also shown an excellent ability to transfer information across domains. However, we have not yet seen a unified framework that combines both transfer learning and representation learning techniques in an end-to-end schema using the knowledge graph.

This research aims to solve the data sparsity and cold start challenges when bringing a recommender system to market. We propose a holistic approach by leveraging knowledge graph, enrich and extract cross-domain information to improve both recommendation model precision and adaptability. The rest of this report is organized as follows: 
First, we went through research questions, objectives and expected outcomes in Section 2. Next, in section 3, we present an extensive literature review of the related work of this study. It will cover current recommender systems, knowledge graph representation and transfer learning research results. Section 4 describes the significance and innovation of this research. Section 5 presents the methodology to conduct this research. Section 6 is about ethics and risk consideration. Finally, Section 7 outlines the entire timeline of this research and reports on the research progress up-to-date.