%!TEX root = main.tex

\section{Introduction and background}
The recommender system is an indispensable technology. It frees us from information overloads (Lu et al. 2015), keeping us focused on information that matters to us the most. Recommender system sometimes also acts as decision assisting agents to simplify or accelerate the decision-making process. Over the years, we see the recommender system being adopted across different industries. Over the past decade, collaborative filtering based approach recommendation algorithm has been at the forefront of recommender systems field. 
The success of collaborative filtering is not only due to its effectiveness. Its simplicity and generic methodology are much appreciated by the industries, despite having known limitation with computation and high requirements on data quality \citep{Amatriain2016}. Data sparsity and constant cold start are common challenges that business face when bringing the recommender system into real-world applications. 

The knowledge graph is a heterogeneous information network. Its ability to retrain rich semantic information and its adaptiveness to dynamic data has been studied in the data mining research \citep{Song2019} extensively. Such characteristics allow the recommender system to unify both items/users side information and user-item interaction inside a single heterogeneous knowledge, thus, effectively improve data density when user-item interactions are sparse. However, so far most graph based recommender system are only able to work on static graphs. The knowledge graph adaptiveness is yet to be further explored, 

Recommender systems exists long before computer was invented. Catalogs can be considered a one of the most primitive content-based recommender systems. In 2006, 100 million datasets were released by Netflix \citep{Bennett2007} The action expedited the research on collaborative filtering (CF). Over a decade, matrix factorization based approach is still at the forefront of recommender systems field. 
% Many new approaches and extensions had been evolved based on collaborative filtering to tackle its computation challenges and data sparsity problems. 
Recent years, transfer learning \citep{Pan2010} techniques has been introduced for improving data sparsity problems. Cross-domain recommender system is an promising area that leverages richer source domain information to complement target domain's recommendation objectives. \citet{Elkahky2015} experimented this approach on multiple Microsoft products, and concluded multi domain recommender system significantly outperforms single domain recommender systems.  

Based on research observation, when bringing a new recommender systems into production, the project is commonly challenged by many aspects. One common early stage challenge is data sparsity. Which, consequently, makes the recommender systems more prone to cold start problems, resulting sub-optimal prediction, and frequent model retrain for adjusting to un-trained data input. In order to ensure a successful project launch, there are 2 areas requires research attention. First, feature enrichment approaches, especially under the condition when data resources is limited. Second, designing an adaptive recommender systems, that is more accommodating to new information, and resilient to cold start scenario. 

In this research, we consider knowledge graph as a heterogeneous information network. Living in a world of information, the data entities around us are mostly interconnected with each other and be formed into an information network. Recent development from \citet{qin2020survey}, \citet{wang2018ripplenet}, and \citet{xi2020graph} have shown a lot of potential from the knowledge graph, which can help in developing generic recommendation framework for solving challenges stated above. Besides algorithm and statistical analytics, our aim is to propose a unified end-to-end framework to ease the barrier of recommender system launch.


This study will focusing on leverage knowledge graph to improve recommender systems prediction capability and results. The rest of this report is organized as follows: 
First we go through research questions, objectives and expected outcomes in Section 2. Next, in section 3, we present an extensive literature review of the related work of this study. It will cover recent recommender systems, knowledge graph representation and transfer learning research results. Section 4 describes the significance and innovation of this research. Section 5 presents the methodology to conduct this research. Section 6 is about the ethics and risk consideration. Finally, Section 7 outlines the entire timeline of this research and reports on the research progress up-to-date.


